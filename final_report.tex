\documentclass[12pt,a4paper]{article}

% --- Packages ---
\usepackage[utf8]{inputenc}
\usepackage[margin=1in]{geometry}
\usepackage{graphicx}
\usepackage{booktabs}
\usepackage{hyperref}
\usepackage{titlesec}
\usepackage{enumitem}
\usepackage{glossaries}
\usepackage[round]{natbib}

% --- Glossary ---
\makeglossaries
\newacronym{LLM}{LLM}{Large Language Models}
\newacronym{RAG}{RAG}{Retrieval-Augmented Generation}
\newacronym{SLM}{SLM}{Small Language Models}

% --- Formatting ---
\hypersetup{
    colorlinks=true,
    linkcolor=blue,
    filecolor=magenta,      
    urlcolor=cyan,
    pdftitle={Final Fellowship Report},
}

\titleformat{\section}{\large\bfseries}{\thesection}{1em}{}[\titlerule]

% --- Metadata ---
\title{Final Research Fellowship Report}
\author{Aqila Farahmand\\University of Urbino Carlo Bo\\\texttt{aqila.farahmand@uniurb.it}}
\date{Reporting Period: March 2025 -- March 2026}

\begin{document}

\maketitle

\begin{abstract}
    This report summarizes the research activities, key findings, and professional development milestones achieved during the one-year fellowship period.
    The work primarily focused on the following research project: Smart Shires: LM-based Intelligent Services Supporting Decentralized and Rural Areas.
    Our work during the fellowship period included the development of MedicoAI, a cross-platform application with offline functionality for the medical question-answering task for patient self-management,
    supporting decentralized and rural areas when the internet connectivity is limited.
    And \gls{RAG}-Enhanced Open \gls{SLM} for the Hypertension Management Chatbots research project that was published in the journal of Medical Systems.

    This report discusses the work done during the fellowship period and the key findings and achievements.

\end{abstract}

\tableofcontents
\newpage

\section{Introduction}\label{sec:introduction}
%
The main goal of our research project studies the feasibility of implementing Large Langauge based intelligent services
supporting decentralized and Rural areas, mainly in the medical domain.
%
I started the fellowship with a comprehensive literature review to identify gaps in existing research and to refine our research questions.
%
\glspl{LLM}, such as HealthLLM~\cite{kim2024healthllmlargelanguagemodels} are valuable in health care, due to their ability to process complex data and generate insights.
However, they come with several challenges, in particular heavy computational requirements and privacy concerns, and some state-of-the-art \glspl{LLM} are not open source and require a paid subscription, which limits their accessibility.
%
Existing \gls{LLM} often rely on cloud-based services, which raises privacy concerns and the risk of information leakage~\cite{zhang2024enablingondevicellmspersonalization}, especially when handling sensitive medical data.
%
Cloud-based \glspl{LLM} also require a stable internet connection, and they may not be suitable for real-time applications due to high latency~\cite{zhang2024enablingondevicellmspersonalization}.
In particular, in rural areas with limited internet connectivity, the use of cloud-based \glspl{LLM} can be challenging.
%
These limitations make \glspl{LLM} less ideal for deployment in decentralized and rural areas, especially in healthcare settings where data privacy is a major concern.

Moreover, existing \glspl{LLM} are often trained on general data and may not perform well in specialized domains like healthcare, where domain-specific knowledge is crucial.
%
To address these challenges, we explored the use of \gls{RAG} together with \glspl{SLM} as a potential solution to enhance the performance of \glspl{SLM}
for specialized tasks such as hypertension management chatbots, while also addressing the limitations of \glspl{LLM} in terms of computational requirements and privacy concerns.
%
In the first few months, I collaborated with my supervisors and other researchers on the \gls{RAG}-Enhanced Open \gls{SLM} for the Hypertension Management Chatbots research project.
We conducted experiments to study the effectiveness of \gls{RAG} techniques in comparison to prompt engineering techniques for enhancing the performance of \glspl{SLM} in the context of hypertension management chatbots.
%
This research project provided valuable insights into using \gls{SLM} as a promising alternative to \glspl{LLM} for specialized tasks.
Our findings show that \gls{RAG} techniques demonstrate equal or better performance compared to \gls{SLM} with prompting techniques alone in terms of medical faithfulness when used for the general medical question-answering tasks.
%
In the second half of the fellowship, to address the limitations of \glspl{LLM} in terms of computational requirements and privacy concerns,
we conducted a literature review to explore the existing approaches for developing \glspl{LLM} that can be deployed on edge devices, such as smartphones and tablets, which are more accessible in decentralized and rural areas.
Based on the insights gained from the literature review, we developed MedicoAI, a prototype application which implements on-device \glspl{SLM} for medical question-answering tasks to support patient self-management in decentralized and rural areas.
Also, the choice of \gls{SLM} over \gls{LLM} was made to address the computational requirements and privacy concerns associated with \gls{LLM},
as \gls{SLM} can be deployed on edge devices without the need for a stable internet connection, and they can be trained on domain-specific data to enhance their performance in specialized tasks.
%
In this study, we investigate the baseline capabilities of \gls{SLM} for general medical question-answering on mobile devices using various prompting techniques. We
specifically selected four state-of-the-art \gls{SLM} with model sizes under 1GB, including Gemma3-1B-IT,
Hammer2.1-0.5b, Qwen2.5-0.5B-Instruct, and Smollm-135M-Instruct.
%
These models are evaluated using two prompt templates, a standard baseline, and one with
medical safety constraints.
Furthermore, we implemented three word-limit configurations to facilitate a fair comparison of token counts and response times across all models.
%
The core design of MedicoAI ensures that user queries and personal data remain on the user’s device, eliminating server-side storage of sensitive medical conversations.
%


\section{Key Results and Achievements}\label{sec:key-results-and-achievements}
Highlight what you accomplished over the 12 months.
\begin{itemize}
    \item \textbf{Deliverable 1:} Conducted the experiments to study the effectiveness of \gls{RAG} techniques in comparison to prompt engineering techniques for enhancing the performance of \glspl{SLM} in the context of hypertension management chatbots.
    \item \textbf{Deliverable 2:} Co-authored the paper which has been published in the journal of Medical Systems~\cite{ragEnhancedSLMs2025}.
    \item \textbf{Deliverable 3:} Developed MedicoAI, a cross-platform application with offline functionality for the medical question-answering task for patient self-management.
    \item \textbf{Deliverable 4:} The paper describing the MedicoAI application experiments is accepted in TELMED workshop at PerCome 2026 for presentation in Pisa, Italy.
\end{itemize}

\section{Professional Development}\label{sec:professional-development}
\begin{itemize}
    \item \textbf{Summer School:} Attended GenAI summer school in Tempera in Finland, which provided me with a comprehensive understanding of the latest advancements in generative AI and its applications in various domains.
    Also, invited to be a part of their panel discussion on the future of Generative AI and its impact on society was a great opportunity to share my insights and learn from other experts in the field.
    \item \textbf{Skills:} Improved my skills in writing scientific paper using LaTeX, Python, \gls{RAG} architecture, \gls{LLM} frameworks, and libraries such as Langchain and HuggingFace.
\end{itemize}

\section{Challenges}
%
One of the main challenges in conducting the research was limited access to datasets for training and evaluating the models, especially in the medical domain where data privacy is a major concern.
This made it difficult to train and evaluate our experiments effectively, and it also limited the scope of our research to some extent.
Also, the lack of standardized benchmarks for evaluating \gls{LLM} in healthcare settings.

\section{Conclusion and Future Outlook}\label{sec:conclusion-and-future-outlook}
%
In conclusion, the research conducted during the fellowship period has provided valuable insights into the use of \gls{RAG} techniques together with \glspl{SLM} for enhancing the performance of language models in specialized tasks, particularly in healthcare settings.
The results of our findings were published in the journal of Medical Systems~\cite{ragEnhancedSLMs2025}.
%
The development of MedicoAI demonstrates the feasibility of deploying \gls{SLM} on edge devices for medical question-answering tasks, which can support patient self-management in decentralized and rural areas
with limited internet connectivity.
The paper describing the MedicoAI application experiments is accepted in TELMED workshop at PerCome 2026 for presentation in Pisa, Italy.
%
This research has implications for the development of intelligent services that can be deployed in decentralized and rural areas, particularly in healthcare settings where data privacy is a major concern.
%

\printglossaries
\bibliographystyle{plainnat}
\bibliography{biblio}

\end{document}